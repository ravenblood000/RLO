\documentclass{article}
\usepackage{amsmath}

\title{CS 250 Course Study Guide}
\author{Sarna Gomasta, Kyla Levin, Alexander Yeung}
\date{Fall 2023}

\usepackage[letterpaper, portrait, margin=1in]{geometry}
\usepackage{graphicx}
\usepackage{enumitem}
\usepackage{ulem}
\usepackage{amssymb}
\usepackage{amsmath}

\begin{document}

\maketitle

\section*{Chapter 1: Sets and Predicate Logic}
This section has the following learning objectives:
\begin{enumerate}
    \item Translate statements to and from propositional logic
    \item Prove statements of propositional logic using truth tables and/or propositional proof rules.
    \item Translate statements to and from predicate logic.
    \item Prove statements of predicate logic using the four quantifier proof rules.
    \item Know and work with the definitions of relations (partial orders and equivalence relations) and properties of functions (one-to-one, onto, bijective).
\end{enumerate}
\subsection*{Q1.1}
Solve the following syllogism. Report the answer as a logical implication, as well as a naturally-spoken English sentence.
\begin{itemize}
    \item The only foods that my doctor recommends are ones that aren't very sweet.
    \item Nothing that agrees with me is good for dinner.
    \item Cake is always very sweet.
    \item My doctor recommends all foods that are good for dinner.
\end{itemize}
\newpage
\subsection*{Answer}
We will use the following variables: $D=\text{doctor recommends it},S=\text{sweet},A=\text{agrees with me},G=\text{good for dinner}$
\begin{itemize}
    \item This can be read as "If my doctor recommends it, then it's not very sweet." $D\rightarrow \neg S$
    \item This can be read as "There is nothing such that if it agrees with me, then it's good for dinner." $A\rightarrow G$
    \item This can be read as "If it's cake, then it's very sweet." $C\rightarrow S$
    \item This can be read as "If it's good for dinner, then my doctor recommends it." $G\rightarrow D$
\end{itemize}
$C\rightarrow S\rightarrow \neg D\rightarrow\neg G$
\\ Transitively, $C\rightarrow\neg G$
\\ In English, "If it's cake, then it's not good for dinner." Or, more naturally, "Cake is not good for dinner."
\newpage
\subsection*{Q1.2}
\begin{enumerate}[label=\alph*.]
    \item What values of $a$ and $b$ make the following statement true? Prove using a truth table.
    $$(a \lor \neg b) \land (b \land \neg a)$$
    \item What values of $a$, $b$, and $c$ make the following statement true? Prove without using a truth table.
    $$(a \lor \neg b) \land (b \land \neg a)\rightarrow (\neg c\land(a \lor b))$$
\end{enumerate}
\newpage
\subsection*{1.2 Answer}
\begin{enumerate}[label=\alph*.]
    \item What values of $a$ and $b$ make the following statement true? Prove using a truth table.
    $$(a \lor \neg b) \land (b \land \neg a)$$
    \begin{tabular}{|c|c|c|c|c|c|c|}
     \hline
     $a$ & $b$ & $\neg a$ & $\neg b$ & $(a\lor \neg b)$ & $(b\land\neg a)$ & $(a\lor \neg b)\land(b\land\neg a)$ \\ 
     \hline
     T & T & F & F & T & F & F \\  
     T & F & F & T & T & F & F \\
     F & T & T & F & F & T & F \\  
     F & F & T & T & T & F & F \\
     \hline
    \end{tabular}
    \\ To fill in the truth table, begin with the primitive values $a$ and $b$ in the leftmost columns. From there, use those values to fill in the successive $\neg a$ and $\neg b$ values in the next two columns. Continue rightways, adding columns with increasingly complex expressions using the truth values from the simpler columns until you have constructed the final expression.
    \\ As a reminder, for the binary operator "OR" ($\lor$), \textit{at least one} of the input needs to be true. For the binary operator "AND" ($\land$), \textit{both} of the input need to be true. And the unary operator "NOT" ($\neg$) will flip the truth value of its input.
    \\ As seen in the final column of the truth table, there is no combination of $a$ and $b$ values that make $(a\lor \neg b)\land(b\land\neg a)$ true.
    \item What values of $a$, $b$, and $c$ make the following statement true? Prove without using a truth table.
    $$(a \lor \neg b) \land (b \land \neg a)\rightarrow (\neg c\land(a\lor b))$$
    \\ All possible values of $a$, $b$, and $c$ will make this statement true. We know from part (a) that the antecedent of the implication step is always false. There are no $a$ and $b$ values that make it true. Therefore, there is no value that the consequent could have to make the whole implication evaluate to true. If the consequent is true, then $F\rightarrow T$ evaluates to true. If the consequent is false, then $F\rightarrow F$ still evaluates to true. 
\end{enumerate}
\newpage
\subsection*{Q1.3}
DEDUCTIVE SEQUENCE PROOF
\subsection*{Answer}
\newpage
\subsection*{1.4}
TODO: Contrapositive or contradiction proof
\subsection*{Answer}
\newpage
\section*{Chapter 2: Quantifiers and Relations}
\subsection*{2.1}
Consider the set $A=\mathbb{N}$
\\ Identify whether the following relations are reflexive, antireflexive, symmetric, antisymmetric, or transitive.
\begin{itemize}
    \item $x$ relates to $y$ if and only if $x=2y$ 
    \item $x$ relates to $y$ if and only if $x\%2=y\%2$
\end{itemize}
\newpage
\subsection*{Answer}
\begin{itemize}
    \item $x$ relates to $y$ if and only if $x=2y$
    \\ This is not reflexive, because for most numbers, $x\neq 2x$. However, it is also not antireflexive, since $x=2x$ is true for $x=0$. It is not symmetric, since if $x=2y$, then $y\neq2x$. The only case when this is true is for $x=y=0$, when $x$ and $y$ are the same. Therefore, this relation is antisymmetric. It is not transitive, since if $x=2y$ and $y=2x$, then $x=4z$, which is not always equal to $2z$.
    \item $x$ relates to $y$ if and only if $x\%2=y\%2$
    \\ This is reflexive, since $x\%2=x\%2$ for all $x$. Because it is reflexive, it cannot be antireflexive. It is symmetric, since if $x\%2=y\%2$, then   $y\%2=x\%2$. It is not antisymmetric, since this can be true of different $x$ and $y$ values. It is transitive, since if $x\%2=y\%2$ and $y\%2=z\%2$, then $x\%2=z\%2$.
\end{itemize}
\newpage
\subsection*{2.2}
TODO: Function question
\subsection*{Answer}
\newpage
\section*{Chapter 3: Number Theory}
\subsection*{Q3.1 General Number Theory}
Prove that for some relatively prime $a$ and $n$, multiplying all numbers in the set $\{1,2,...,n-2,n-1\}$ by $a$ will output a permutation of $\{1,2,...,n-2,n-1\}$.
\\ Hint: There are two parts to this proof---existence and uniqueness. To show that multiplication by $a$ permutes the elements of the set, you must show that the multiplication always produces an element in the set (existence) and you must show that that element will never be repeatedly produced via multiplication by $a$ (uniqueness). Consider how these relate to modular arithmetic and modular inverses.
\newpage
\subsection*{Answer}
As stated in the hint, we must show both the existence and uniqueness of the elements from the set when multiplied by $a$.
\\ First, to prove existence, we will prove that for some relatively prime $a$ and $n$, and for some elements $b,c\in \{1,2,...,n-2,n-1\}$,  $ab=c$ (mod $n$), unless $b=0$. This is equivalent to saying that "If $ab=0$ (mod $n$), then $b=0$ (mod $n$)". Since $a$ and $n$ are relatively prime, $a$ has an inverse in mod $n$. Multiply both sides by $a^{-1}$ (mod $n$) to get $aba^{-1}=0(a^{-1})$ (mod $n$). $aa^{-1}$ will cancel out to 1, showing that $b=0$ (mod $n$).
\\ Next, to prove uniqueness, we will show that for some relatively prime $a$ and $n$, if $ab=ac$ (mod $n$), then $b=c$. Once again, because $a$ and $n$ are relatively prime, we know $a$ must have some inverse $a^{-1}$ (mod $n$). Multiplying both sides by $a^{-1}$ gives $aba^{-1}=aca^{-1}$ (mod $n$), leaving $b=c$ (mod $n$).
\\ Putting the two pieces together, we know that multiplying any element of $\{1,2,...,n-2,n-1\}$ by some constant $a$, which is relatively prime to $n$, will never result in 0 and therefore will always produce some number in the range of 1 to $n-1$. Furthermore, no multiplication will result in the same number. Therefore, the output will be a permutation of the original set $\{1,2,...,n-2,n-1\}$.
\newpage
\subsection*{Q3.2 Euclid's Algorithm}

Let $a = 473$ and $b = 47$. Run Euclid's Algorithm on a and b to find their GCD. What can you conclude about a and b from your run of Euclid's Algorithm?

\subsection*{Answer}

\begin{align*}
473 &= 10 \times 47 + 3 \\
47 &= 15 \times 3 + 2 \\
3 &= 1 \times 2 + 1 \\
2 &= 2 \times 1 + 0 \\
\end{align*}

The GCD of 473 and 47 is 1 since the algorithm ends with a remainder of 1. Since their GCD is 1, the two numbers are relatively prime.
\newpage
\subsection*{Chinese Remainder Theorem (CRT)}
\newpage
\section*{Chapter 4: Induction and Recursion}
\subsection*{4.1}
Explain the error in the following induction proof:
\\ \textbf{Claim}: All students are the same height.
\\ Base case: Consider a set of students of size 1. Any student is the same height as themselves, and so all students in the set are the same height.
\\ \textbf{Inductive hypothesis}: All students in a set of size $n$ have the same height. 
\\ \textbf{Inductive step}: Now suppose we have a set of $n+1$ students, labeled $s_1$ through $s_{n+1}$: $\{s_1,s_2,s_3,...,s_{n-1},s_n,s_{n+1}\}$.
\\ Consider the entire set of $n+1$ students as two individual sets of $n$ students each.
\\ The first set contains students $s_1$ through $s_n$: $\{s_1,s_2,s_3,...,s_{n-1},s_n\}$
\\ The second set contains students $s_2$ through $s_{n+1}$: $\{s_2,s_3,...,s_{n-1},s_n,s_{n+1}\}$
\\ Both sets contain $n$ students, and so by the inductive hypothesis, all students in each set must be of the same height. And if all students in each set have the same height, then all students in the full $n+1$ set must have the same height.
\\ In conclusion, all students in any set of size $n\geq1$ must have the same height.
\newpage
\subsection*{Answer}
The error in the proof can be attributed to the overlap between the two sets in the inductive step. The base case is easy to observe for some set $\{s_1\}$, but when the inductive step is applied to "step" from $n=1$ to $n=2$, the two sets described in the inductive step would be $\{s_1\}$ and $\{s_2\}$, in which may be two students of differing heights. So while $s_1$ and $s_2$ are the same heights as themselves, this does not mean that they are the same height as one another.
\newpage
\subsection*{4.2}
Suppose we have a recursively-defined function (much like a Fibonacci sequence) $f_n=5f_{n-1}-6f_{n-2}$ where $f_0=2$ and $f_1=5$.
\\ Prove inductively that $f_n=2^n+3^n$ for all $n\geq0$.
\subsection*{Answer}
\textbf{Claim}: $f_n=2^n+3^n$ for all $n\geq0$
\\ \textbf{Base case}: For $n=0$: $f_0=2$, and $2^0+3^0=1+1=2$. For $n=1$: $f_1=5$ and $2^1+3^1=2+3=5$.
\\ \textbf{Inductive hypothesis}: Assume that $f_n=2^n+3^n$ for all $n\geq0$
\\ \textbf{Inductive step}:
\begin{itemize}[label=]
    \item $f_{n+1}=5f_n-6f_{n-1}$
    \item $f_{n+1}=5(2^n+3^n)-6(2^{n-1}+3^{n-1})$ By inductive hypothesis
    \item $f_{n+1}=5(2^n)+5(3^n)-3(2^n)-2(3^n)$
    \item $f_{n+1}=2(2^n)+3(3^n)$
    \item $f_{n+1}=2^{n+1}+3^{n+1}$
\end{itemize}
\newpage
\section*{Chapter 9: Graphs and Trees}
\newpage
\section*{Chapter 5: Regular Expressions}
\subsection*{5.1}
Assume the alphabet $\sum=\{0,1\}$. If the English description is given, provide the regular expression. If the regular expression is given, give a short, English description of the accepted set of strings.
\begin{enumerate}[label=\alph*.]
    \item The set of strings that contain at least one 0 and at least one 1.
    \item $(1+\lambda)(00^*1)^*0^*$
    \item The set of strings such that all pairs of adjacent 0's appear before any pairs of adjacent 1's.
\end{enumerate}
\subsection*{Answer}
\begin{enumerate}[label=\alph*.]
    \item This can be thought of as either a 0 appearing before the 1, or the 1 appearing before the 0. Since these are the only mandated characters, the rest of the string can be any combination of 0's and 1's that come before, in-between, or after the 0 and 1. In other words, there are two possible string formats: $(0+1)^*0(0+1)^*1(0+1)^*$ or $(0+1)^*1(0+1)^*0(0+1)^*$
    \\ Together, the full regex is $((0+1)^*0(0+1)^*1(0+1)^*)+((0+1)^*1(0+1)^*0(0+1)^*)$
    \item The language starts with either a 1 or an empty string. From there, it is followed by a 0, some number of zero's, and then a 1. Just looking at these parts, the string can start as either:
    \\ 1 0 0... 1 
    \\ 0 0... 1
    \\ Even if the 0... is omitted, the string can only start with 0 1. 1 1 is not allowed.
    \\ Looking further, we can observe that within the $(00^*1)$ part, removing the optional 0's, $(01)$ the structure mandates that 0 must come before 1. So if this block repeats, the structure will looks like this: 0 1 0 1 with more 0's padding the in-between. And since only more 0's are added onto the end, we can conclude that this will be the set of all strings without consecutive 1's.
    \item This can be divided into two parts: The first part of the string that does not allow 11's, but does allow 00's, and the second part of the string that does not allow 00's, but does allow 11's.
    \\ For a binary string where 11 cannot occur, this could be any combination of 0's or 10's: $(0+10)^*$
    \\ No 11's: $(0+10)^*(1+\lambda)$
    \\ No 00's: $(1+01)^*(0+\lambda)$
\end{enumerate}
\newpage
\section*{Chapter 14: Finite Automata}
\newpage
\section*{Chapter 15: Formal Language Theory}

\end{document}